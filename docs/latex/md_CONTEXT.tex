Each module of the pipeline implement various callbacks that parse and transform an {\bfseries{o\+Z\+::\+H\+T\+T\+P\+::\+Request}} to create an {\bfseries{o\+Z\+::\+H\+T\+T\+P\+::\+Response}}. Thus, the pipeline need a way to store each information though the various modules\+: {\bfseries{o\+Z\+::\+Context}}.

\subsection*{Context creation}

The context can be default created but to make it process-\/able, you must specify a {\bfseries{o\+Z\+::\+Packet}} containing a client endpoint and its message buffer. At this point, you can run the {\bfseries{o\+Z\+::\+Pipeline\+::run\+Pipeline}} function to process the context and create a response out of it. 
\begin{DoxyCode}{0}
\DoxyCodeLine{ \{C++\}}
\DoxyCodeLine{// Callback triggered when receiving messages from clients on default HTTP port}
\DoxyCodeLine{void MyServer::onReceiveNetworkPacket(oZ::Packet \&\&packet)}
\DoxyCodeLine{\{}
\DoxyCodeLine{    // Create a context out of a network Packet}
\DoxyCodeLine{    oZ::Context context(std::move(packet));}
\DoxyCodeLine{}
\DoxyCodeLine{    // Run the pipeline with this new context and send its response to the client}
\DoxyCodeLine{    \_pipeline.runPipeline(context);}
\DoxyCodeLine{    sendResponseToClient(context);}
\DoxyCodeLine{\}}
\end{DoxyCode}


\subsection*{Encryption}

As you may know, H\+T\+TP uses default port 80. However, if you want to add encryption to your server exchanges, you must use the default H\+T\+T\+PS port 443. Thus, the context need a way to store an encryption flag (to know if the client uses H\+T\+T\+PS) and an encryption key. This data is contained in the {\bfseries{o\+Z\+::\+Packet}} class.

If you use open\+S\+SL for the encryption, you may need to use {\bfseries{Pipeline\+::on\+Connection}}, {\bfseries{Pipeline\+::on\+Disconnection}}, {\bfseries{I\+Module\+::on\+Connection}} and {\bfseries{I\+Module\+::on\+Disconnection}} to implement it as a standalone module.


\begin{DoxyCode}{0}
\DoxyCodeLine{ \{C++\}}
\DoxyCodeLine{// Callback triggered when receiving messages from clients on default HTTPS port}
\DoxyCodeLine{void MyServer::onReceiveEncryptedNetworkPacket(oZ::Packet \&\&packet)}
\DoxyCodeLine{\{}
\DoxyCodeLine{    // Create a context out of a network Packet}
\DoxyCodeLine{    oZ::Context context(std::move(packet));}
\DoxyCodeLine{}
\DoxyCodeLine{    // Set the encryption flag in the context}
\DoxyCodeLine{    context.getPacket().setEncryption(true);}
\DoxyCodeLine{    // Run the pipeline with this new context and send its response to the client}
\DoxyCodeLine{    \_pipeline.runPipeline(context);}
\DoxyCodeLine{    sendResponseToClient(context);}
\DoxyCodeLine{\}}
\end{DoxyCode}


\subsection*{Cache handling}

In this section we will see how caching can be achieved if you really wish to go for performances. The {\bfseries{o\+Z\+::\+Context}} class has a {\bfseries{o\+Z\+::\+Context\+::is\+Constant}} function that tells if the current context is constant, and thus can be cached. If some of your modules aren\textquotesingle{}t supporting cache for specific H\+T\+TP methods, you can use the {\bfseries{o\+Z\+::\+Context\+::not\+Constant}} function to set the constant state to false. It\textquotesingle{}s up to you to implement caching. You can accomplish in by creating a module or by handling it directly in the server. 